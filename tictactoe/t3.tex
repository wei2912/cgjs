\documentclass{article}

\usepackage{amsmath}
\usepackage{algorithm}
\usepackage{algpseudocode}

\newcommand{\var}[1]{\mathit{#1}}
\newcommand{\qt}[1]{\mbox{`#1'}}
\algrenewcommand{\algorithmicreturn}{\State \textbf{return}}

\title{Week 3: Tic-Tac-Toe}
\author{Benjamin Tan and Ng Wei En}
\date{\today}

\begin{document}
\maketitle

\section{Introduction}

Tic-Tac-Toe is a paper-and-pencil game for two players, X and O, who take turns marking the spaces in a 3 x 3 grid. The player who succeeds in placing three respective marks in a horizontal, vertical, or diagonal row wins the game.

\section{Representing a game board}

A 3 x 3 array of markings is used. Markings can be either `X', `O' or null, the last one indicating that no marking has been placed.

For a game board of:

\begin{verbatim}
O O X
_ O _
X X X
\end{verbatim}

where `\_' represents an empty square and `O' and `X' represent their respective players, the board is represented as the following array:

\begin{verbatim}
[[`O', `O', `X'],
 [null, `O', null],
 [`X', `X', `X']]
\end{verbatim}

\section{Checking for a win}

The first part of our algorithm obtains all sequences of markings which, if equal to either `X' or `O', constitutes as a win for that player. Given the following grid:

\begin{verbatim}
X O X
O X O
X _ _
\end{verbatim}

The following sequences of markings should be obtained:

\begin{verbatim}
// Rows
X O X
O X O
X _ _

// Columns
X O X
O X _
X O _

// Forward Diagonal
X X X
// Backward Diagonal
X X _
\end{verbatim}

In the example, the forward diagonal is comprised of only `X'. Hence, it can be concluded that the player `X' has won.

We iterate through all these sequences.

For each sequence, if every element in the sequence is equal to either `X' or `O', it can be concluded that either `X' or `O' has won. Hence, the player that has won is returned. Otherwise, the function continues to look through the list of sequences.

If no winning sequence is found, the function returns null, which indicates that the game has not finished.

\section{Checking for a draw}

To simplify things, we will only declare a draw when the board is fully filled.

\end{document}
