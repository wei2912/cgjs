\documentclass[a4paper]{article}

\usepackage[english]{babel}
\usepackage[utf8]{inputenc}
\usepackage{amsmath}
\usepackage{graphicx}
\usepackage[]{algorithm2e}

\title{W2: Tic-Tac-Toe}

\author{Benjamin Tan & Wei En}

\date{\today}

\begin{document}
\maketitle

\section{Introduction}

Tic-Tac-Toe is a paper-and-pencil game for two players, X and O, who take turns marking the spaces in a 3 x 3 grid. The player who succeeds in placing three respective marks in a horizontal, vertical, or diagonal row wins the game.

\section{Function API}

We need to create an \emph{Application Programming Interface} for our code. The \emph{Application Programming Interface} is the interface of our functions that the code of our user interface will call. Without a proper interface, our user interface will not work properly.

We define the following objects:

\begin{description}
  \item[Grid] Stores a grid of markings.
  \item[Game] Contains the state of the game.
\end{description}


\subsection{Grid}

The Grid object is a two-dimensional array of markings.

Markings can be either `X', `O' or null, the last one indicating that no marking has been placed.


\subsection{Game}

The Game object contains the following properties and functions:

\begin{description}
  \item[Game] constructor for the Game object.

  \item[grid] Instance of a Grid object.
  \item[curTurn] Current turn.

  \item[isWin()] Check if any player has won.
  \item[move(pos)] Make the next move.
\end{description}

\subsection{Game}

This function is a special function which creates a new object.

\subsubsection{grid}

grid is an instance of a Grid object.

\subsubsection{curTurn}

curTurn marks the current turn for the player. Its value can be either 'X' or 'O'. The first player to go is 'X'.

\subsubsection{isWin()}

isWin checks if the player has already won. We use the following logic to find out if the player has already won:

\begin{algorithm}
\uIf{isWin'(O)}{
  \Return O
}
\uElseIf{isWin'(X)}{
  \Return X
}
\Else{
  \Return null
}
\end{algorithm}

\end{document}