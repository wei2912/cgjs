\documentclass{article}

\usepackage{amsmath}
\usepackage{algorithm}
\usepackage{algpseudocode}

\usepackage{mathtools}
\DeclarePairedDelimiter{\floor}{\lfloor}{\rfloor}

\newcommand{\var}[1]{\mathit{#1}}
\newcommand{\qt}[1]{\mbox{`#1'}}
\algrenewcommand{\algorithmicreturn}{\State \textbf{return}}

\title{Week 2: Guess the Number}
\author{Benjamin Tan and Ng Wei En}
\date{\today}

\begin{document}
\maketitle

\section{Introduction}

Guess the Number is a game where the player gives a guess to a number chosen by the computer and the computer either states that the number is correct, or gives a lower or upper bound on the number.

In this document, we detail out the API of your Tic Tac Toe implementation in what is known as `psuedocode'.

\section{API}

We define the following objects:

\begin{description}
  \item[GuessTheNum] Contains the state of the Guess the Number game.
\end{description}

\subsection{Game}

The `GuessTheNum' object contains the following properties and functions:

\begin{description}
  \item[GuessTheNum()] Constructor for the GuessTheNum object.

  \item[num] Computer-generated number.

  \item[checkNum()] Check a number given.
  \item[genNum()] Generate a new number.
\end{description}

\subsubsection{GuessTheNum()}

`GuessTheNum()' is the constructor of the Game object. It initializes `num'.

\subsubsection{num}

`num' is a number generated by the computer at the start of the game. It is initialized when a `GuessTheNum' object is created through the use of `genNum()'.

\subsubsection{checkNum()}

`checkNum()` is a function which compares a user-chosen number with `num'. The function returns a number less than 0 if the user-chosen number is less than `num', a number greater than 0 if the user-chosen number is more than `num' and 0 if both are equal.

The function is very trivial to write:

\begin{algorithm}
\caption{Compare a user-chosen number with the computer generated number.}

\begin{algorithmic}[1]
\Procedure{checkNum}{x}
  \If{$x < this.num$}
    \Return $-1$
  \ElsIf{$x = this.num$}
    \Return $0$
  \ElsIf{$x > this.num$}
    \Return $1$
  \EndIf
\EndProcedure
\end{algorithmic}
\end{algorithm}

\subsubsection{genNum()}

`genNum()' is a function which returns a randomly generated number within the range [1, 100].

\begin{algorithm}
\caption{Generate a random number within the range [1, 100].}

\begin{algorithmic}[1]
\Procedure{genNum}{}
  \Return $\floor{\Call{random}{} * 100} + 1$
\EndProcedure
\end{algorithmic}
\end{algorithm}

The function `random' generates a random number within the range [0, 1). Multiplying this value expands the range to [0, 100).

The floor function, as denoted by $\floor{x}$, obtains the largest integer that is smaller than $x$. Numbers such as 4.32 will be rounded down to 4.

This brings our range to [0, 99]. Adding 1 gives us [1, 100], which is our desired range.

\end{document}
