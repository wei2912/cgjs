\documentclass{article}

\usepackage{amsmath}
\usepackage{algorithm}
\usepackage{algpseudocode}

\usepackage{mathtools}
\DeclarePairedDelimiter{\floor}{\lfloor}{\rfloor}

\newcommand{\var}[1]{\mathit{#1}}
\newcommand{\qt}[1]{\mbox{`#1'}}
\algrenewcommand{\algorithmicreturn}{\State \textbf{return}}

\title{Week 2: Guess the Number}
\author{Benjamin Tan and Ng Wei En}
\date{\today}

\begin{document}
\maketitle

\section{Introduction}

Guess the Number is a game where the player gives a guess to a number chosen by the computer and the computer either states that the number is correct, or gives a lower or upper bound on the number.

\section{Random Number Generation}

`genNum()' is a function which returns a randomly generated number within the range [1, 100].

\begin{algorithm}
\caption{Generate a random number within the range [1, 100].}

\begin{algorithmic}[1]
\Procedure{genNum}{}
  \Return $\ceil{\Call{random}{} * 100}$
\EndProcedure
\end{algorithmic}
\end{algorithm}

The function `random' generates a random number within the range [0, 1). Multiplying this by 100 expands the range to [0, 100).

The floor function, as denoted by $\floor{x}$, returns the largest integer that is smaller than $x$. Numbers such as 4.32 will be rounded down to 4.

On the other hand, the ceiling function, as denoted by $\ceil{x}$, returns the smallest integer that is larger than $x$. Numbers such as 2.62 will be rounded up to 3.

This brings our range to [1, 100], which is our desired range.

\end{document}
